% !TeX root = ../main.tex

\chapter{工作总结与展望}

\section{工作总结}
本文探讨了事件相机这一新型视觉设备在火焰检测领域的应用前景,构建了基于事件相机的火焰检测算法模型。

1. 将事件相机引入火焰检测领域,利用事件相机拍摄了不同场景的火焰图像,同时也拍摄了多类的非火焰干扰图像,例如行人、灯光等,制作了一个较完整的基于事件相机的火焰检测数据集,可适用于后续基于事件相机的研究工作。
 
2. 基于建立的火焰检测数据集,将数据转化为与传统视觉算法相适配的二维平面二值图像,采用基于事件数据的思路对火焰的静态特征和动态特征进行了提取与研究,并结合事件数据的独特特点,引入了一些新的火焰特征参数,可以适用于此后的基于事件相机的火焰检测工作。

3. 基于火焰检测数据集和对火焰特征的提取工作,采用机器学习的思路,利用支持向量机建立了火焰检测算法模型,对投入训练的火焰特征参数的选取、相关模型参数的确定进行了优化,经过训练后的模型可以较好地处理火焰的二分类问题。训练后的火焰检测模型平均精确度可达80\%,构建与训练时间总和不超过5s,可以适用于后续的基于事件相机的火焰检测。

\section{创新点}
1. 引入事件相机这一新型视觉设备进行火焰检测研究,拍摄制作了一个较为完整的、包括火焰场景和多种类的非火焰干扰场景的火焰事件数据集。数据集可供之后长期使用。

2. 对于事件数据这一新颖的数据类型,采用不同于传统计算机视觉处理的思路,使其转化为适配传统视觉算法的形式,并结合事件数据的独特性,引入或重新对一些火焰特征参数进行定义。

3. 针对事件数据建立了基于支持向量机的火焰检测算法模型,可以读取拍摄的火焰事件数据,有效处理场景内火焰二分类问题,训练后的火焰检测模型平均精确度可达80\%,构建与训练时间总和不超过5s。
\section{工作展望}
本文基于事件相机建立的数据集与检测算法模型,仍然存在以下的改进可能:

1.本次拍摄的火焰场景为木垛火场景,后续希望对更多的火焰场景进行补充,例如其他类型或者可燃物材质
的火焰、多片火焰的场景,同理非火场景也可以进行更多完善,加入更多不同的可能干扰源。

2.本次研究火焰特征时由于事件相机的本身光谱特点等原因,未对火焰的颜色、纹理进行研究。两者
均为火焰的重要特征,后续希望结合传统RGB图像对其进行研究与补充。

3.本次利用机器学习的思路进行检测模型的构建,之后的工作希望采用深度学习的思路,利用深度神经网络
对模型进行训练,同时将上述方法与传统视频图像检测方法一起进行横向的比较和评估。