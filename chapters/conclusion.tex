% !TeX root = ../main.tex

\chapter{工作总结与展望}

\section{工作总结}
 本文探讨了事件相机这一新型视觉设备在火焰检测领域的应用前景,主要工作集中在,对于传统的火焰识别
 二分类问题:

 1.将事件相机引入火焰识别领域,利用事件相机拍摄了大量的火焰图像,以事件这种新颖的数据类型来储存火焰信息,
 能够以不同于传统视觉处理的方法对数据进行处理与还原, 同时也拍摄了多类的非火焰干扰场景,例如行人,灯光,
 并针对火焰识别领域事件数据集的缺乏,利用上述拍摄内容制作了一个较完整的基于事件相机的火焰检测数据集,可适用于后续
 基于事件相机的相关工作。
 
 2.基于建立的火焰事件数据集,采用不同于传统计算机视觉处理的思路,能够在最大程度保持数据信息原始性的基础上
 将数据转化为与传统视觉算法相适配的二维平面二值图像,同时采用基于事件数据的思路,对火焰的静态特征和动态特征
 进行了提取与研究,并对其中一些特征做了进一步的分析。此外,还结合事件数据的独特特点,引入了一些新的自定义的火焰特征参数,
 可以适用于此后的基于事件相机的火焰检测工作。

 3.基于数据集和对火焰特征的相关提取和研究工作,采用机器学习的思路,利用支持向量机,建立火焰检测算法模型,
 经过多次的实验不断进行改进,对投入训练的火焰特征参数的选取,相关模型参数的确定进行了优化,经过训练后的
 模型可以较好地处理火焰的二分类问题,可以适用于后续的基于事件相机的火焰检测,说明了事件相机在火焰识别领域
 未来可期,亟待后续的持续探索。

\section{创新点}
 1.在目前火焰检测数据集多为可见光相机拍摄的情况下,引入事件相机拍摄制作了一个较为完整的
 火焰事件数据集。数据集包括火焰场景和多种类的非火焰干扰场景,可供之后长期使用或进行更多的补充完善。

 2.对于事件数据这一新颖的数据类型,采用不同于传统计算机视觉处理的思路,使其转化为适配
 传统视觉算法的形式。并结合事件数据的独特性,引入或重新对一些火焰特征参数进行定义,可在
 以后的工作中沿用。

 3.针对事件数据建立了基于SVM的火焰检测算法模型,可以读取拍摄的火焰事件数据,有效处理场景内火焰二分类问题。

\section{工作展望}
本文基于事件相机建立的数据集与检测算法模型,仍然存在以下的改进可能:

1.本次拍摄的火焰场景为木垛火场景,后续希望对更多的火焰场景进行补充,例如其他类型或者可燃物材质
的火焰,多片火焰的场景,同理非火场景也可以进行更多完善,加入更多不同的可能干扰源。

2.本次研究火焰特征时由于事件相机的本身光谱特点等原因,未对火焰的颜色,纹理进行研究。两者
均为火焰的重要特征,后续我们希望结合传统RGB图像对其进行研究与补充。

3.本次利用机器学习的思路进行检测模型的构建,之后我们希望采用深度学习的思路,利用深度神经网络
对模型进行训练,同时将上述方法与传统视频图像检测方法一起进行横向的比较和评估。