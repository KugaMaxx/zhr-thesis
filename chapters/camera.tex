\chapter{相关理论概述}
1970 年 Fukushima 等人在对生物视网膜结构的研究工作中,展示了首个电子视网膜离散模型,这种新颖的
获取信息的方式很快引起了相关领域学者们的兴趣,但是由于陌生的异步逻辑领域以及像素响应特征的不均匀性,
事件相机的发展一直处处受限,一直到2008年首部事件相机(DVS)问世,也是世界上第一台商用事件相机。此后,在经
过了十数年的发展,事件相机已经开发出了多种版本款式,应用针对不同的工作任务与场景,如ATIS,DAVIS,CeleX,
本次我们引入工作的是型号为DAVIS346的事件相机,后续都会在这款相机的基础上进行工作的开展,后续不再进行累述。

\section{事件相机}
\subsection{工作原理}
我们在前面也粗略地提起到了事件相机与传统可见光相机在原理与机制上的巨大不同。传统的可见光传感设备
为一次成像,图象是以一个固定的帧率被设备输出,每次成像过程中,相机的各个相素会进行电荷收集并在曝光
流程结束后将像素上的电荷信息转化为数字信号输出至外界。这种工作原理下,拍摄高速运动物体时缺陷就
极为明显,图像很容易就会出现模糊与失真。事件相机就是这样一部为了解决该问题的,仿生物视网膜结构模型的
传感设备,它由生物视网膜细胞具有对亮度瞬变的高敏感性获得启发,每一个像素独立地,仅会在对数亮度值的变化超出阈值时
进行一次输出。这种机制,给予了这类传感器更广的动态范围,也自动消除了环境中非变化物体等产生的冗杂信息。

事件相机内部主要工作模块的电路逻辑图如图所示,主要由差分电路,感光器和比较仪三部分组成,实现对生物视网膜结构的
模拟。对于生物而言,感受到光线后,它视网膜上的受体细胞受到相应的刺激,就会自发将光信息转变成神经信息,后段细胞还会分别筛选出亮部和暗部,经过
神经节细胞的处理后,信息就传递到了大脑皮层,呈现为视觉。而在事件相机中,光信息在感光器电路中会转换成电信号,它会通过放大器,最终由比较器根据
亮度变化进行分离,将光信息转化为“事件”信息,最后在经过一系列的后续处理后,转为图像输出。

假定亮度为I,事件相机中的亮度定义都是其对数值,即L=log(I),那么在某一个时刻,某一个像素点处的亮度改变就可以记为

其中,是一个极小的时间间隔,如果亮度的变化超出了相机所设定的阈值C,事件就会触发,这个过程也可以表示为

其中,表示极性,也就是亮度发生了正向(ON)或者负向(OFF)变化,我们的事件数据就以这样一个形式输出出来,
可以看到,我们后续工作中所提起的“事件”,往往就是这样的一个四维向量(空间坐标,时间坐标,极性)。

\subsection{对事件数据的处理}
由上面我们的叙述可见,事件数据是一种新颖的数据储存类型,对其的处理和使用方式也是使用事件相机的核心。
其中一种常用方法便是根据事件的空间和时间坐标将其一一映射为3D点云,这里,如图所示,我们展示了这种“事件”最为直观
的视觉效果,取一段时间范围内所有的事件点云(其中颜色是对极性的表示),将其全部映射堆叠后,可以很直观地看出,
成功地呈现了图像的视觉效果。这里值得一提的是,3D事件点云是不满足旋转不变性假设的,事件之间在时空上是存在着因果联系。

另一种方法是舍弃掉时间维度,从而将三维空间转换为二维图像,这种方法会为传统的图像处理算法带来极大便利。
我们可以每次按照一定的时间间隔提取一定数量的事件,类似传统的可见光传感器,按照前后次序定义帧数,从而也可以得到一帧帧的图象,
如图所示,从而实现对传统算法的适配。

此外还有一些很独特,针对性较强的方法,例如将事件作为脉冲信号,送入相应的网络,这里不再详细介绍。这些独特的方法
也在一些特定的场景发挥着优秀作用,为事件相机应用带来了更多契机。

\subsection{事件相机的独特优势}
基于上述我们介绍的事件相机的独特机制,事件相机相对于传统可见光传感器有着下列的先天优势:

1.高时间分辨率。由于各相素独立工作的机制,省去了传统可见光相机的很多中间流程,可以达到微秒级分辨率,几乎不会出现
运动模糊。

2.低数据量。每个像素当且仅当变化超出阈值才会输出一次,很多无用的干扰信息都会被机制自动剔除,因此事件相机的输出数据远远
低于传统设备,在处理速度等很多方面就有着很大优势。

3.低功耗。由于2的原因以及事件相机不需要模数转换器读取像素,后续的处理功耗也自然较低,往往是毫瓦级,与传统设备有百倍以上的差距。

4.高动态范围。由于采用对数制光强,事件相机往往可以达到传统相机2倍以上动态范围,可以适应很多欠曝和过曝场景。

\section{本章小结}
本章我们详细介绍了事件相机的工作原理,介绍了它的独特机制和数据形式,介绍了应用工作中对事件数据的一些常见处理方法,
并且介绍了其相对于传统相机的几个显著优点。后面的章节,我们将利用事件相机开展并介绍本次的主要工作进程。