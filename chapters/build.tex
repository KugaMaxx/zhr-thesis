% !TeX root = ../main.tex

\chapter{基于事件相机的火焰数据集构建}
在之前绪论的火焰数据集部分,我们已经提到目前领域内常用的数据集以可见光相机拍摄为主,缺乏效果
优秀的,公开的基于事件相机的火焰数据集,本章中我们将使用DAVIS346事件相机收集相关的火焰数据并
着手制作一个初步的火焰检测数据集,后面我们统一用FIRE指代称呼。

\section{火焰数据集}
\subsection{数据集概述}
火焰检测数据集和其他的目标检测相似,也分为图象与视频两个大类,分别用于静态与动态不同的检测算法
训练与测试。例如著名的Dyntex公开库中,提供了数百个动态视频纹理系列,其中的每个序列时间大于十秒,一秒25帧,
均为720x576的图像,同时每个均备注了一些相关的信息,包括采集的时间,设备,环境条件等。这里我们值得提及的是,视频数据集作为训练动态算法的连续帧视频文件,其帧数可以有所不同,由几秒到几分钟不等,但是往往
需要包含有火和无火两类场景,不同的环境条件,例如光照,风况,也均会对其效果造成不小影响。

同时目前也出现了大量的火焰烟雾场景为主的数据集,期望能够对早期烟雾进行识别而更早确定火灾场景。这里再以MIVIA数据集为例,它分火焰集与烟雾集,包含的
火灾集视频序列共31个,分别是17个无火焰视频和14个火灾场景,均为不同环境拍摄,分类较为细致。同时它的烟雾集设置了很多远镜头烟雾场景,内含天空、云、强自然光等干扰,很考验
待测算法的准确度。此外还有中国科学技术大学火灾科学国家实验室制作的烟雾数据集,已经过人工标注,其中包含上万张的烟雾与非烟雾图像,包括真实烟雾与合成烟雾,
可以有效胜任深度网络训练。

\subsection{数据集制作}
我们使用型号DAVIS346的事件相机,模拟火灾场景的监控视角,其中可燃物使用小块木垛堆叠进行火实验,
包括多种不同材质,最终经过筛选后,汇总了共x个序列,其中含有x条单火焰序列,x条双火焰序列,作为正例存在,
剩余的和上述总量数量接近的x条无火焰序列作为反例补充。为了保证这个数据集的效果以及泛用性,在拍摄过程中,我们对
拍摄的角度进行了多次的变换,同时也涵盖了火焰发展的前中后期各个阶段,我们还尝试加入了一些干扰性的环境条件或干扰,
例如有人员经过,暂时性的遮挡等。

这里我们展示了一些拍摄片段的视觉效果,如图x所示,同时对序列列表展示,如图x所示

\subsection{数据集标定}
火焰检测数据集的使用需要真实标签数据作为真实参考结果来与检测算法所预测的标签数据进行对比,这里我们为FIRE数据集制作真实标签
是通过人工手动标定完成的,最大限度地保证结果的有效性。在标定过程中,我们对每个序列以33ms的时间间隔进行提取,定义为帧,将序列
分割完成二维平面上的投影,这就作为我们的事件帧,接着又使用Dark Label对每个序列逐帧标注火焰,最终我们将APS图像标注结果和事件帧
标注结果进行融合给出最终结果,如图中例子所示

\section{本章小结}
本章我们主要介绍了我们本次工作所使用的数据集的主要内容,制作过程与方法,后面的章节我们就会在这个FIRE数据集基础上
进行提取特征工作以及检测算法构建工作。