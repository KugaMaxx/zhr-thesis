% !TeX root = ../main.tex

\begin{acknowledgements}

    转瞬四年,忽然而已。停笔那刻,忽然就想起了那句“欲买桂花同载酒,终不似,少年游”,再去回望四年前,曾经清晰的一幕幕,如今已变得似乎飘渺而不可触,曾经年少的执念与痴狂,也多在岁月的洗礼下烟消云散,付之一笑。时间,确实是最锋利的刃,没有什么能让它的轨迹产生哪怕一毫的偏移,它却改变了一切。

    回忆这四年,欢亦有,悲亦有,迷茫与成熟共舞,保守与改变同在。无论是刚入校时对基础科学的热忱,还是现在更多去思考与面对现实的焦虑,亦或是最初自以为挣脱枷锁,扶摇直上的一腔热情,亦可能是最终发觉不过仍为笼中鸟的满目彷徨,都已经是一道道留下的足迹。站在这里回忆过去,何尝不是在岁月的长河中刻舟求剑呢?可以感慨,允许叹息,但它们终究留在我的身后,无法也不必回首。

    不必回首的是过往,需要铭记的是同行的人。但是首先,我想要感谢我自己,人生路漫漫,绝大多数人,都不过路途中暂时的过客,自己,才是最长久,最坚定的同伴。我曾踏足山巅,也曾跌落低谷,二者都使自己受益良多,感谢自己的那份默默坚持,感谢自己对内心那种持之以恒的信任。

    接着,我想要对我的指导老师,宋卫国老师致以最由衷的感谢。宋老师是火灾科学领域的学术领军人物,我很有幸,火灾学入门课程是由宋老师亲自授课,也曾在宋老师的多个课题组辗转,参与了多个实验并完成了一个课题。在我提笔写下这段文字的时候,两年前第一次略带紧张去见宋老师的情景仍然历历在目,宋老师是一位面相严肃实则极其亲和的导师,他严谨的学术精神,深厚的专业造诣,从那时起就一直让我钦佩。他是我的科研引路人,更是我想要成为的榜样。

    我要对我遇到的师兄师姐们表示由衷感谢。何卓洋师兄,一位非常优秀的师兄,郭奖的获得者,作为我的课程助教,他将迷惘于原专业的我引入了火灾实验室这一平台,并为我答疑解惑,介绍老师。在他身上,“谦谦君子,温文如玉”仿佛得到了最好的诠释。余杭师兄,我第一次参与实验室具体实验时由他指导,无论是实验步骤的详细手操介绍还是答疑,亦或是后面实验不小心犯错后他帮我的补救,彻底消散了我一个新人初来乍到时的惶恐。丁赛喆师兄,指导我时间最久的师兄,在我初窥科研时事无巨细地为我指导解惑。大到长期规划,小至具体代码bug,他总能提出自己的一份见解,一句建议。我似乎永远可以在他这里学到东西,他就像一口不见底的古井,遇到问题他就是最渊博的老师,已然数不清自己在他这里学到了多少。从未有过不耐烦,永远对我保持着热情,他对我而言,亦兄,亦师,亦友。此外还有很多帮助过我的师兄师姐,难以历数,再次由衷感谢。

    我由衷感谢自己的父母。他们这些年来从未在行动上反对自己的决定与选择,永远以朋友的身份进行支持与建议。他们是最能照顾我情绪的人,没有之一,他们也在苍老,但是他们让我时刻清楚一个事实,无论如何,家永远是一个爱的港湾。

    我由衷感谢我的同学朋友,他们有的就在身边,有的身处全球各地,但是我们的联系从未中断,分享与倾听永远是不变的旋律。我衷心感谢科大的每一位老师,他们的传授与言行,从各方各面塑造了现在的我。我衷心感谢这四年每一个相遇的人,可能是宿舍阿姨,可能是活动搭档,可能是一面之缘的陌生人,人生如逆旅,我亦是行人,我们的相遇就是这段旅程最美好的意义。

    行文至此,突然有些临表涕零的感觉。作为一个理工科生,自己自然也算不上极其感性的人,但是,就好像突然一个人站在门口大声告诉你,“这段旅程结束了,该离开了”的时候,你一时没回过神,“是不是有些潦草呢?”。确实,掰着手指,我相信可以数出上百处缺憾。但是,假如真的给你一次重来的机会呢?你保证真的不会再说出同样的话吗?我不能。我一直很讨厌那些展望未来的长篇大论,因为在我看来,立足当下才是最积极的态度,听过这样一个故事,一只年轻的鱼告诉年老的鱼:“我受不了这片水了,我很想找到叫大海的东西!”年老的鱼讲:“大海?你一直就在海里啊。”大海再广大,也是由点点滴滴的水组成,而水,就是我们的日复一日,习以为常的生活。
    
    所以,既往不恋,当下不杂,未来不迎。
\end{acknowledgements}
