% !TeX root = ../main.tex

\ustcsetup{
  keywords  = {事件相机, 火焰检测算法, 机器学习, 火灾探测},
  keywords* = {Event Camera, Flame Detection Algorithm, Machine Learning,Fire Detection},
}

\begin{abstract}
  事件相机是一种新型的、模拟生物视网膜原理的视觉传感设备,以完全不同于标准相机的方式工作,基于事件驱动的方式来捕捉场景中的动态变化。相对于传统相机的每一帧图像,事件相机只会记录下物体的运动变化信息,这一特点赋予了它在很多高速、高变化频率场景下的优秀信息处理能力,可以广泛应用到多个研究领域中。火焰检测是一个十分契合事件相机工作特点的场景,理想情况下可以通过事件相机实现对火焰全天候的监控,并利用其独特的特性排除无关背景噪音干扰,提高大幅火焰检测效率。事件相机在火灾领域的应用刚刚起步。

  本文旨在探讨事件相机在火灾探测领域的可行性与应用价值,建立起较为完整的检测算法框架。首先,使用事件相机对多种不同材质的木垛火进行多角度拍摄,同时拍摄一定数量的无火焰场景作为对照,构建了初步的火焰数据集。其次,对火焰数据集中火焰的静态与动态特征进行分析,建立了较完整的事件数据特征提取算法框架。最后,采用机器学习的思路,利用支持向量机训练构建了针对火焰事件的检测算法模型,并将模型的结果进行了可视化展示呈现。
  
  本文最终搭建的基于事件相机的火焰检测算法模型,可接收一定长度的火焰事件数据片段,提取其中的相关静态与动态特征,经过一系列预处理,最终返回模型的检测结果,即是否存在火焰以及包含火焰的检测框位置坐标。利用评价算法对该模型的检测精确率、召回率等参数进行了客观的评估,结果表明该模型能够较好地处理火焰的传统二分类问题。
\end{abstract}

\begin{abstract*}
  An event camera is a new type of vision sensor that mimics the principles of the biological retina, operating in a manner completely different from standard cameras. Instead of capturing static images frame by frame, it is based on an event-driven approach to record dynamic changes in a scene. Unlike traditional cameras that capture entire frames, event cameras only record changes in the movement of objects. This characteristic endows them with excellent information processing capabilities in many high-speed, high-frequency change scenarios, making them applicable in various research fields.
  Fire detection is a scenario that perfectly aligns with the working principles of event cameras. Ideally, event cameras can achieve all-weather monitoring of flames and utilize their unique properties to eliminate irrelevant background noise interference, significantly improving fire detection efficiency. The application of event cameras in the field of fire detection is just beginning.

  This paper aims to explore the feasibility and application value of event cameras in the field of fire detection, establishing a relatively comprehensive detection algorithm framework. Firstly, an event camera was used to capture multi-angle shots of woodpile fires with various materials, while also capturing a certain number of flame-free scenes as controls, thus creating an initial flame dataset. Secondly, the static and dynamic characteristics of flames in the dataset were analyzed, and a comprehensive event data feature extraction algorithm framework was established. Finally, using a machine learning approach, a flame event detection algorithm model was constructed and trained using support vector machines, and the results of the model were visually presented.

  The final flame detection algorithm model based on the event camera, constructed in this paper, can receive flame event data segments of a certain length, extract relevant static and dynamic features, and after a series of preprocessing steps, return the model's detection results. These results indicate whether a flame is present and provide the coordinates of the detection bounding box containing the flame. The evaluation algorithm objectively assessed parameters such as the model's detection accuracy and recall rate. The results indicate that this model can effectively handle the traditional binary classification problem of flame detection.







\end{abstract*}
